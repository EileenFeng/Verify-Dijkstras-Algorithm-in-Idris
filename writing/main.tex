% Credits are indicated where needed. The general idea is based on a template by Vel (vel@LaTeXTemplates.com) and Frits Wenneker.

\documentclass[12pt, a4paper]{article} % General settings in the beginning (defines the document class of your paper)
% 11pt = is the font size
% A4 is the paper size
% “article” is your document class

%----------------------------------------------------------------------------------------
% Packages
%----------------------------------------------------------------------------------------

% Necessary
\usepackage[german,english]{babel} % English and German language 
\usepackage{booktabs} % Horizontal rules in tables 
% For generating tables, use “LaTeX” online generator (https://www.tablesgenerator.com)
\usepackage{comment} % Necessary to comment several paragraphs at once
\usepackage[utf8]{inputenc} % Required for international characters
\usepackage[T1]{fontenc} % Required for output font encoding for international characters

% Might be helpful
\usepackage{amsmath,amsfonts,amsthm} % Math packages which might be useful for equations
\usepackage{tikz} % For tikz figures (to draw arrow diagrams, see a guide how to use them)
\usepackage{tikz-cd}
\usetikzlibrary{positioning,arrows} % Adding libraries for arrows
\usetikzlibrary{decorations.pathreplacing} % Adding libraries for decorations and paths
\usepackage{tikzsymbols} % For amazing symbols ;) https://mirror.hmc.edu/ctan/graphics/pgf/contrib/tikzsymbols/tikzsymbols.pdf 
\usepackage{blindtext} % To add some blind text in your paper


%---------------------------------------------------------------------------------
% Additional settings
%---------------------------------------------------------------------------------

%---------------------------------------------------------------------------------
% Define your margins
\usepackage{geometry} % Necessary package for defining margins

\geometry{
  top=2cm, % Defines top margin
  bottom=2cm, % Defines bottom margin
  left=2.2cm, % Defines left margin
  right=2.2cm, % Defines right margin
  includehead, % Includes space for a header
  %includefoot, % Includes space for a footer
  %showframe, % Uncomment if you want to show how it looks on the page 
}

\setlength{\parindent}{15pt} % Adjust to set you indent globally 

%---------------------------------------------------------------------------------
% Define your spacing
\usepackage{setspace} % Required for spacing
% Two options:
\linespread{1.3}
%\onehalfspacing % one-half-spacing linespread

%----------------------------------------------------------------------------------------
% Define your fonts
\usepackage[T1]{fontenc} % Output font encoding for international characters
\usepackage[utf8]{inputenc} % Required for inputting international characters

\usepackage{XCharter} % Use the XCharter font


%---------------------------------------------------------------------------------
% Define your headers and footers

\usepackage{fancyhdr} % Package is needed to define header and footer
\pagestyle{fancy} % Allows you to customize the headers and footers

%\renewcommand{\sectionmark}[1]{\markboth{#1}{}} % Removes the section number from the header when \leftmark is used

% Headers
\lhead{} % Define left header
\chead{\textit{}} % Define center header - e.g. add your paper title
\rhead{} % Define right header

% Footers
\lfoot{} % Define left footer
\cfoot{\footnotesize \thepage} % Define center footer
\rfoot{ } % Define right footer

%---------------------------------------------------------------------------------
% Add information on bibliography
\usepackage{natbib} % Use natbib for citing
\usepackage{har2nat} % Allows to use harvard package with natbib https://mirror.reismil.ch/CTAN/macros/latex/contrib/har2nat/har2nat.pdf

% For citing with natbib, you may want to use this reference sheet: 
% http://merkel.texture.rocks/Latex/natbib.php

%---------------------------------------------------------------------------------
% Add field for signature (Reference: https://tex.stackexchange.com/questions/35942/how-to-create-a-signature-date-page)
\newcommand{\signature}[2][5cm]{%
  \begin{tabular}{@{}p{#1}@{}}
    #2 \\[2\normalbaselineskip] \hrule \\[0pt]
    {\small \textit{Signature}} \\[2\normalbaselineskip] \hrule \\[0pt]
    {\small \textit{Place, Date}}
  \end{tabular}
}
%---------------------------------------------------------------------------------
% General information
%---------------------------------------------------------------------------------
\title{Shortest Path Algorithms Verification with Idris} % Adds your title
\author{
Yazhe Feng % Add your first and last name
    %\thanks{} % Adds a footnote to your title
    %\institution{YOUR INSTITUTION} % Adds your institution
  }

\date{\small \today} % Adds the current date to your “cover” page; leave empty if you do not want to add a date


%---------------------------------------------------------------------------------
% Define what’s in your document
%---------------------------------------------------------------------------------

\begin{document}


% If you want a cover page, uncomment "%---------------------------------------------------------------------------------
% Cover page
%---------------------------------------------------------------------------------

% Here are more templates for other cover pages: https://www.latextemplates.com/cat/title-pages

% This example is based on this cover page example: https://www.latextemplates.com/template/academic-title-page

\begin{titlepage} % Starts new environment where the page number is not displayed and the count starts at 1 for the next page

%------------------------------------------------
%	Institutional information
%------------------------------------------------
	
\begin{minipage}{0.4\textwidth} % Begins new environment (like a text box)
    \begin{flushleft} % Sets environment on the left side of the paper
    \large
    University of XX\\ % Add your institution
    Chair of Political Science IV\\ % Add the chair
    Fall 2018\\ % Add term
    COURSE TITLE\\ % Add course title
    Supervisor: NAME % Add instructor/supervisor name 
    \end{flushleft}
\end{minipage}
	
\vspace*{2in} % Adds some space in-between
	
\center % Centre everything on the page

%------------------------------------------------
%	Main part
%------------------------------------------------
	
{\huge\bfseries TITLE OF YOUR PAPER}\\[0.4cm] % Add your paper title 
{\large\today}\\[0.4cm] % Add date (current day)
FIRSTNAME LASTNAME % Add your name
	
\vfill % Adds additional space

%------------------------------------------------
%	General information about the author
%------------------------------------------------

\vfill % Adds additional space

Your contact info \\ % Add your contact info
Your Program \\ % Add info about your program
Semester you are enrolled \\ % Add info about your semester

\vfill % Adds additional space

%------------------------------------------------
%	Word count
%------------------------------------------------

\vfill % Adds additional space
	
Word count: XXXX % To indicate the word count
% How to check words in a LaTeX document: https://www.overleaf.com/help/85-is-there-a-way-to-run-a-word-count-that-doesnt-include-latex-commands
	

	
\end{titlepage}" and uncomment "\begin{comment}" and "\end{comment}" to comment the following lines
%%---------------------------------------------------------------------------------
% Cover page
%---------------------------------------------------------------------------------

% Here are more templates for other cover pages: https://www.latextemplates.com/cat/title-pages

% This example is based on this cover page example: https://www.latextemplates.com/template/academic-title-page

\begin{titlepage} % Starts new environment where the page number is not displayed and the count starts at 1 for the next page

%------------------------------------------------
%	Institutional information
%------------------------------------------------
	
\begin{minipage}{0.4\textwidth} % Begins new environment (like a text box)
    \begin{flushleft} % Sets environment on the left side of the paper
    \large
    University of XX\\ % Add your institution
    Chair of Political Science IV\\ % Add the chair
    Fall 2018\\ % Add term
    COURSE TITLE\\ % Add course title
    Supervisor: NAME % Add instructor/supervisor name 
    \end{flushleft}
\end{minipage}
	
\vspace*{2in} % Adds some space in-between
	
\center % Centre everything on the page

%------------------------------------------------
%	Main part
%------------------------------------------------
	
{\huge\bfseries TITLE OF YOUR PAPER}\\[0.4cm] % Add your paper title 
{\large\today}\\[0.4cm] % Add date (current day)
FIRSTNAME LASTNAME % Add your name
	
\vfill % Adds additional space

%------------------------------------------------
%	General information about the author
%------------------------------------------------

\vfill % Adds additional space

Your contact info \\ % Add your contact info
Your Program \\ % Add info about your program
Semester you are enrolled \\ % Add info about your semester

\vfill % Adds additional space

%------------------------------------------------
%	Word count
%------------------------------------------------

\vfill % Adds additional space
	
Word count: XXXX % To indicate the word count
% How to check words in a LaTeX document: https://www.overleaf.com/help/85-is-there-a-way-to-run-a-word-count-that-doesnt-include-latex-commands
	

	
\end{titlepage}

%\begin{comment}
\maketitle % Print your title, author name and date; comment if you want a cover page 

% \begin{center} % Center text
%     Word count: XXXX
% % How to check words in a LaTeX document: https://www.overleaf.com/help/85-is-there-a-way-to-run-a-word-count-that-doesnt-include-latex-commands
% \end{center}
% %\end{comment}

%----------------------------------------------------------------------------------------
% Introduction
%----------------------------------------------------------------------------------------
\setcounter{page}{1} % Sets counter of page to 1

\section{Introduction} % Add a section title
Shortest path problems deal with finding the path with minimum distance value between two nodes in a given graph. One variation of shortest path problem is single-source shortest path problem, which focus on finding the path with minimum distance value from one source to all other vertices within the graph. Among all the algorithms that solve single-source shortest path problems, Dijkstra's and Bellman-Ford algorithms are the most renowned, and are implemented by software concerning various fields in real-life applications, such as finding the shortest path in road map, or routing path with minimum cost in networks. 

Existing resource on verifying programs that implement Dijkstra's and Bellman-Ford are relatively limited. In most cases the correctness of program relies on the theoretical proof of the underlying algorithm, whereas the verification of program itself remains unattended. Consider the increasing significance of software implementing both algorithms in solving real-world issues, it is important to be able to verify the behaviors and ensure the correctness of these software. 

In this work we focuses on verifying simple programs that implement Dijstra's and Bellman-Ford algorithms. Our implementaion starts with defining data structures used in both algorithms(for instance node, edge, and graph), and involves proving properties of data types such as natural numbers and list. We aim to present verification as a programming issue, showing how properties of data types, behaviors of functions, and correctness of programs can be verified not only through theoretical proofs, but also through implementations.

The structure of the paper is as follows. Section 2 describes the significance and value of algorithm verification, and the reason of choosing Idris as the language for verifying programs. Section 3 provides some background on Dijkstra's and Bellman-Ford algorithms, and also briefly introduce the Idris programming language. Section 4 includes an overview of our program, including pseudocode and theoretical proofs of Dijkstra's and Bellman-Ford, which serves as the guildeline for our implementation. In section 5 we will cover more details of our programs, including type signature, function definitions, and key lemmas, presenting a detailed demonstration of our verification program. 

% \subsection{Citing} % Add another subsection
% Citing in \LaTeX is easy. You could easier cite with the text flow like this ``Referring to \citet{collier2004greed} ...''  or at the end of the sentence \cite{collier2004greed}. You can also cite pages like this \citep[55]{collier2004greed}. If you want to add an additional note, you might want to do it this way \citep[cp.][22]{collier2004greed} or like this \citep[cp.][]{collier2004greed}.\\
% \blindtext % Adds some blintext to your text

%----------------------------------------------------------------------------------------
% Literature review
%----------------------------------------------------------------------------------------
\section{Motivation}

\section{Background}
\subsection{Dijkstra's Algorithm}



\section{High-level Contribution}


\section{Literature review}


\subsection{Some subsection}

\subsubsection{And a subsubsection}

%---------------------------------------------------------------------------------
% Theory
%---------------------------------------------------------------------------------

\section{Theory}



%----------------------------------------------------------------------------------------
% Research design
%----------------------------------------------------------------------------------------

\section{Research Design}



%----------------------------------------------------------------------------------------
% Analysis
%----------------------------------------------------------------------------------------

\section{Analysis}



%----------------------------------------------------------------------------------------
% Conclusion
%----------------------------------------------------------------------------------------

\section{Conclusion}

%----------------------------------------------------------------------------------------
% Bibliography
%----------------------------------------------------------------------------------------
\newpage % Includes a new page

\pagenumbering{roman} % Changes page numbering to roman page numbers
%\bibliography{literature}

\bibliography{literature.bib} % Add the filename of your bibliography
\bibliographystyle{apsr} % Defines your bibliography style

% For citing, please see this sheet: http://merkel.texture.rocks/Latex/natbib.php

%----------------------------------------------------------------------------------------
% Appendix
%----------------------------------------------------------------------------------------
\newpage % Includes a new page
\section*{Appendix} % Stars disable section numbers
% \appendix % Uncomment if you want to add an "automatic" appendix
\pagenumbering{Roman} % Changes page numbering to Roman page numbers


%----------------------------------------------------------------------------------------
% Declaration
%----------------------------------------------------------------------------------------
\newpage % Includes a page break
\thispagestyle{empty} % Leaves the page style empty (no page number, no header, no footer)
\section*{Statutory Declaration} % Stars disable section numbers

\vspace*{1in} % Adds extra space between two paragraphs


%\vspace*{1in} % Adds extra space

% % Add field for signature, date, and place
% \hfill \signature{} 


%---------------------------------------------------------------------------------

\end{document}
