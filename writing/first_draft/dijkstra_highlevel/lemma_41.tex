\begin{sublemma}
Given any two nodes $v, w$, the prefix of the shortest path $\Delta(v, w)$ is also a shortest path. 
\end{sublemma}
\begin{proof}
We will prove Lemma 3.1 by contradiction. 
\\
Consider any node $q$ in the sequence of $\Delta(v, w)$, we have $\Delta(v, w) = ve_0v_0e_1v_2...v_i q v_j....v_{n-1}e_nw$. Suppose the prefix of $\Delta(v, w)$ from $v$ to $q$, denote as $p(v, q)$, is not the shortest path from $v$ to $q$. Then we know $p(v, q) = ve_0v_0e_1v_2...v_iq$ is a path from $v$ to $q$ and $length(p(v, q)) > length(\Delta(v, q))$. 
\\
Based on the definition of shortest path, we know: 
\\\\
\ftab $length(\Delta(v, w)) \leq length(p), \forall  p \in path(v, w)$
\\\\
Fenote the path after the node $q$ as $p(q, w) = q v_j....v_{n-1}e_nw$, since $\Delta(v, w) = ve_0v_0e_1v_2...v_i q v_j....v_{n-1}e_nw$, then $\Delta(v, w) = p(v, q) + p(q, w)$, and that $length(\Delta(v, w)) = length(p(v, q)) + length(p(q, w))$. Then we have: 
\\\\
\tab$length(\Delta(v, w)) = length(p(v, q)) + length(p(q, w)) \leq length(p), \forall p \in path(v, w)$
\\\\
Since $p(v, q)$ is not the shortest path from $v$ to $q$ by assumption, then based on the definition of shortest path, $length(p(v, q)) < length(\Delta(v, w))$. Hence there exists another $v-w$ path $p'(v, w)$ such that: 
\\\\
\ftab $p'(v, w) \in path(v, w)$\\
\ftab $p'(v, w) = \Delta(v, q) + p(q, w)$ \\ 
\ftab $length(p'(v, w)) = length(\Delta(v, q)) + length(p(q, w))$ \\ 
\ftab\tab\tab\htab\tsp$< length(p(v, q)) + length(p(q, w))$ \\
\ftab i.e. $length(p'(v, w)) < length(\Delta(v, w))$
\\\\
Hence we have reached a contradiction. Thus by the principle of prove by contradiction, for any the prefix $p(v, q)$ of $\Delta(v, w)$ is the shortest path from $v$ to $q$. Lemma 3.1 holds. 
\end{proof}
\tab \\