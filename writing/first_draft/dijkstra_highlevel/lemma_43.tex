%-----------------------------------------------------------------------------------------------
% Dijkstra's lemma 4.3
%-----------------------------------------------------------------------------------------------
\begin{sublemma}
For any node $v \in g$, if after the $i^{th}$ iteration, $dist_{i+1}[v] = \delta(v)$, then for each proceeding $j^{th}$ iteration, $j > i$, $dist_{j+1}[v] = dist_{i+1}[v] = \delta(v)$. 
\end{sublemma}
\begin{proof}
We will prove \texttt{Lemma 4.3} by induction on the number iterations after the $i^{th}$ iteration. 
\\
Let P(n) be: For any node $v \in g$, if after the $i^{th}$ iteration, $dist_{i+1}[v] = \delta(v)$, then for the $(i+n)^{th}$ iteration, $n \geq 1$, $dist_{i+n+1}[v] = dist_{i+1}[v] = \delta(v)$

\paragraph*{Base Case}: We shall show P(1) holds. 
\\
During the $(i+1)^{th}$ iteration, suppose $u_{i+1}$ is the node being explored, then $dist_{i+2}[v]$ is calculated as: 
\\\\
\tab\[
        dist_{i+2}[v] = \left.
       \begin{cases} 
          min(dist_{i+1}[v], dist_{i+1}[u_{i+1}] + weight(u_{i+1}, v)), & (u_{i+1}, v)) \in g \\ 
          dist_{i+1}[v] & otherwise 
        \end{cases}
        \right\}
      \]
\\\\
If $(u_{i+1}, v)) \in g $, then if $dist_{i+1}[u_{i+1}]$ is the length of some $s-u_{i+1}$ path, then $(dist_{i+1}[u_{i+1}] + weight(u_{i+1}, v))$ is the length of some $s-v$ path. Since $dist_{i+1}[v] = \delta(v)$, then based on the definition of shortest path, $dist_{i+1}[v] \leq dist_{i+1}[u_{i+1}] + weight(u_{i+1}, v)$, and hence $dist_{i+2}[v] = dist_{i+1}[v] = \delta(v)$. 
\\
If $u_{i+1}$ does not have an edge to $v$, then $dist_{i+2}[v] = dist_{i+1}[v] = \delta(v)$. 
\\
Hence in either cases, $dist_{i+2}[v] = dist_{i+1}[v] = \delta(v)$. P(1) holds. 


\paragraph*{Inductive Hypothesis}: Suppose P(k) holds, that is, if after the $i^{th}$ iteration, $dist_{i+1}[v] = \delta(v)$, then for the $(i+k)^{th}$ iteration, $n \geq 1$, $dist_{i+k+1}[v] = dist_{i+1}[v] = \delta(v)$. 


\paragraph*{Inductive Step}: We shall show P(k+1) holds. 
\\
For the node $u_{i+k+1}$ being explored during the $(i+k+1)^{th}$ iteration, there are two cases: (1) $(u_{i+k+1}, v) \in g$; (2) $u_{i+k+1}$ does not have an edge to $v$. We will show that P(k+1) holds under both cases separately. 
\\
\textbf{Case 1: $(u_{i+k+1}, v) \in g$}
\\
If $u_{i+k+1}$ has an edge to $v$, then based on the algorithm, for $dist_{i+k+2}[v]$, we have: 
\\\\
  \tab $dist_{i+k+2}[v] = min(dist_{i+k+1}[v], dist_{i+k+1}[u_{i+k+1}] + weight(u_{i+k+1}, v))$
\\\\
Since based on our inductive hypothesis, $dist_{i+k+1}[v] = dist_{i+1}[v] = \delta(v)$, then if $dist_{i+k+1}[u_{i+k+1}] $ is the length of some $s-u_{i+k+1}$ path, then $(dist_{i+k+1}[u_{i+1}] + weight(u_{i+k+1}, v))$ is the length of some $s-v$ path, and hence $dist_{i+k+1}[v] = \delta(v) \leq (dist_{i+k+1}[u_{i+1}] + weight(u_{i+k+1}, v))$. Then: 
\\\\
 \tab $dist_{i+k+2}[v] = min(dist_{i+k+1}[v], dist_{i+k+1}[u_{i+k+1}] + weight(u_{i+k+1}, v)) \\
 \tab\tab\tab = dist_{i+k+1}[v]\\
 \tab\tab\tab = dist_{i+1}[v] = \delta(v)$
\\\\
P(k+1) holds under \texttt{Case 1}. 
\\
\textbf{Case 2: $u_{i+k+1}$ does not have an edge to $v$}
\\
Since $u_{i+k+1}$ does not have an edge to $v$, then $dist_{i+k+2}[v] = dist_{i+k+1}[v]$. Based on the inductive hypothesis, $dist_{i+k+1}[v] = dist_{i+1}[v] = \delta(v)$. then $dist_{i+k+2}[v] = dist_{i+1}[v] = \delta(v)$. P(k+1) holds for \texttt{Case (2)}. 
\\
Thus P(k+1) holds. By the principle of prove by induction, P(n) holds. \texttt{Lemma 4.3} proved. 

\end{proof}
