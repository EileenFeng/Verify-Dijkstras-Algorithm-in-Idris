
\section{Related Work}
Existing work on verifying Dijkstra's algorithm is relatively limited, and few resources are found for the verification of Bellman-Ford algorithm. Robin Mange and Jonathan Kuhn demonstrates an implementation that verifies Dijkstra's algorithm with the Jahob verification system in their report on efficient proving of Java implementations[3]. Although few resource has been found on the concrete implementation of this work, the report illustrates that as Jahob allows programmers to provide specification of their function's behaviors in high-level logic(HOL), program verification can be reduced to the problem of the validity of HOL formulas. 
\\

Klasen et. al. from the University of Koblenz and Landau present a concrete implementation of Dijkstra's algorithm in Java and its verification with the KeY system[1]. The verification process involves specifying the behavior of each function with preconditions, postconditions, and invariants, and the KeY system checks a function as correct with respect to its specifications if the postconditions hold after execution. Jean-Christophe Filliâtre, a senior researcher from the National Center for Scientific Research(CNRS), offers an implementation of Dijkstra's algorithm along with its verification in Why3, a deductive program verification platform that relies on external theorem provers[4][5]. Both verifications above are largely dependent on theorem proving systems. Unlike Filliâtre and Klasen et. al., our work relies on a significantly smaller trusted code base, indicating that considerable amount of proofs will be presented explicitly in our implementation rather than provided by external theorem provers.
