
Verifying the correctness of programs is important, however in most real-life applications, the correctness of software is never verified directly, rather, it relies on the correctness of the algorithms it implements. This raises an issue concerning the gap between the expected and actual behavior of programs, that theoretical proof of algorithms can never validate the actual behavior of programs. The significance and value of verification, therefore, lies on the fact that it allows us to verify programs themselves rather than the algorithms behind them. 
\\ 

Dijkstra's and Bellman-Ford algorithms are two of the most renowned and widely-applied shortest path algorithms, however existing resource on verifying both algorithms are relatively limited. In this thesis, we offer verifications for the implementations of both algorithm. In additional, we aim to present verification as a programming issue. We want to show that with certain programming languages, verifying the correctness of programs can be achieved with type checking, that if the program's correctness is not guaranteed, then our verification program will fail to be type checked.
\\

Based on the above motivations, the Idris programming language is chosen over other verification tools and proof management systems. Idris is a functional programming language with dependent types, which allows programmers to provide more specification on function's behaviors in its type signature. As we plan to achieve verification with type checking, this feature of Idris can be significantly helpful as often times it is important to establish tight connection between functions and its input data in a verification program. In addition, Idris's compiler-supported interactive editing feature provides precise description of functions' behaviors according to their types, allowing programmer to use types as guidance for writing program, which offers considerable assistance during our implementation. Section 3 covers more backgrounds on the Idris programming language. 