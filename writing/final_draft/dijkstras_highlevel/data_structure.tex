\subsection{Data Structures}
Dijkstra's algorithm requires non-negative edge weights and valid input graph, and the data structures in our implementation are designed to ensure these properties of input values. An overview of data structures in our implementation is presented below, and a detailed description is provided under Section 5. 
\\

Denote \texttt{gsize} as the size of graph, i.e. the number of vertices in a graph. A graph $g$ is defined as a vector containing $\texttt{gsize}$ number of adjacent lists, one for each node in the graph, and a node is defined as a data structure carrying a value of type \texttt{Fin gsize}. An adjacent list for a node $n \in g$ is defined as a list of tuples $(n', edge_w)$, where the first element $n'$ in each tuple is a neighbor of $n$ in $g$, and the second element $edge_w$ is the weight of the edge $(n, n')$ in $g$. To access the adjacent list for a particularly node, the \texttt{Fin gsize} type value carried by this node is used to index the graph $g$. As the graph is defined as a vector of length \texttt{gsize}, the definition of node data type ensures that every well-typed node is a valid vertex in the graph, and that each indexing to the graph data structure are guaranteed to be in-bound.
\\

The type of edge weight is user-defined in our implementation. Specifically, we define a \texttt{WeightOps} data type, which carries a user-specified type for the edge weight, along with operators and properties proofs for this type, which includes arithmetic operators, proof of non-negative value, and proof of plus associativity. As all edge weight are non-negative, and we assume a connected input graph, all edge weight should be non-negative and not equal infinity, whereas Dijkstra's algorithm initialize the distance value of all nodes in the graph (except the source node) as infinity. Based on this consideration we defined a \inl{Distance} data type in addition to the user-defined edge weight type. \inl{Distance} is parameterized over the user-defined weight type and can have value of either infinity, or the sum of edge weights. 

% \subsection{Assumptions}
% Our implementation is based on the following assumptions. 
% \begin{enumerate}
%   \item Weight of edges are positive
%   \item Distance value can only be zero, infinity, or summation of edge weights
%   \item All nodes $n$ and edge $e$ are valid: $n, e \in g$
% \end{enumerate}
