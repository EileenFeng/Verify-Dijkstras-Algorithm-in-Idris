%-----------------------------------------------------------------------------------------------
% Dijkstra's correctness proof
%-----------------------------------------------------------------------------------------------
\subsection{Proof of Correctness} \label{mproof}
This section provides a mathematical proof for our Dijkstra's implementation, which includes proof of program termination and proof of correct program behavior.

\subsubsection{Lemmas} \label{lemmasM}
\tab\\
Denote $explored$ as the list of nodes in $g$ but not in $unexplored$, i.e., $explored$ stored all nodes whose neighbors have been updated by the algorithm. We index $explored$ by the number of iterations, such that $explored_i$ denotes the value of $explored$ at the beginning of the $i^{th}$ iteration.
\tab\\
\import{./}{lemma_41}
\import{./}{lemma_42}
\import{./}{lemma_43}
\import{./}{lemma_44}
\import{./}{lemma_45}
{}
\subsubsection{Proof of Termination} 
\begin{proof}{}
The inner for loop is guaranteed to terminate as the algorithm goes through each adjacent node exactly once. As the size of list \texttt{unexplored} decreases by one during each iteration of the while loop, the algorithm is guaranteed to terminate. 
\end{proof}

\subsubsection{Prove of Correctness}
\begin{proof}
By applying \texttt{Lemma 4.5} to the last iteration, denote as $m^{th}$ iteration, of the algorithm, we obtained that for all nodes $n$ in the explored list, $dist_{m+1}[n]$ is indeed the shortest path distance value from source $s$ to $n$, hence Dijkstra's algorithm indeed calculates the shortest path distance value from the source $s$ to each node $n \in g$. 
\end{proof}