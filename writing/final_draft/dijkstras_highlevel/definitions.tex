
%-----------------------------------------------------------------------------------------------
% Dijkstra's concept definition
%-----------------------------------------------------------------------------------------------
\subsection{Definition} \label{definitions}
Our implementation and correctness proof are based on the following definitions of key concepts used in Dijkstra's algorithm. The following definitions are inspired by Epp \cite{discrete}.
\\

\theoremstyle{definition}
\begin{definition}\textbf{Path}\\
A path from node $v_0$ to $v_n$ is a finite sequence of adjacent vertices of G, which does not contain any repeated edge or vertex. A path from $v$ to $w$ has the form: 
\begin{center}
 $v_0v_1....v_{n-1}v_n$ 
\end{center}
where each adjacent nodes $v_{i-1}, v_i$ has an edge from $v_{i-1}$ to $v_i$ in G. We denote the set of paths from $v_0$ to $v_n$ as $path(v_0, v_n)$.
\end{definition}
\tab
\begin{definition}\textbf{Prefix of Path}\\
Given a path $p$ from node $v_0$ to $v_n$, i.e., $p = v_0v_1....v_{n-1}v_n \in path(v_0, v_n)$, the prefix of this $v_0 - v_n$ path is defined as a subsequence of $p$ that starts with $v_0$ and ends with a node $v_i$ for some $0 \leq i \leq n$ ($v_i$ is a vertex in the nodes sequence of $p$). 
\end{definition}
\tab
\begin{definition}\textbf{Length of Path} \\
The length of a path $p = v_0v_1....v_{n-1}v_n$ is the sum of the weights of all edges in $p$. We write: 
\begin{center}
  $length(p) = \sum weight(v_{i-1}, v_i), \forall v_{i-1}, v_i \in p$ where $(v_{i-1}, v_i) \in G$. 
\end{center} 
\end{definition}
\tab
\begin{definition}\textbf{Shortest Path}\\
Denote $\Delta(s, v)$ as an arbitrary choice of a shortest path from $s$ to $v$, and denote $\delta(v)$ as the length of $\Delta(s, v)$. $\Delta(s, v)$ must fulfills: 
\begin{center}
$\Delta(s, v) \in path(s, v)$ 
\\
and 
\\
$\forall p' \in path(s, v)$, $length(\Delta(s, v)) = \delta(v) \leq length(p')$
\end{center}
\end{definition}