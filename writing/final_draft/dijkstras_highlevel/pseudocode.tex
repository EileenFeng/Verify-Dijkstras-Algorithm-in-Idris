
%-----------------------------------------------------------------------------------------------
% Dijkstra's pseudocode
%-----------------------------------------------------------------------------------------------
\subsection{Pseudocode} \label{pseudo}
We denote $(u, v)$ as an edge from node $u$ to $v$, $weight(u, v)$ as the weight of edge $(u, v)$. If $(u, v)$ is not an edge in the graph, then define $weight(u, v) = \infty$. Let \texttt{gsize} denote the size of the input graph, i.e., the number of nodes in the graph. The type \texttt{Graph gsize weight} specifies a graph with \texttt{gsize} nodes and edge weight of type \texttt{weight}. 
\\
Given input graph $g$ and source node $s$ with types:
\begin{lstlisting}
      g : Graph gsize weight
      s : Node gsize
\end{lstlisting}

Define $unexplored$ as the set of unexplored nodes, and $dist$ as the set of distance values\footnote{For convenience purpose, in this thesis we denotes the `distance value' for a node `n' in a graph `g' as the distance from the source node to n in `g'} from $s$ to all nodes in $g$ calculated by the Dijkstra's algorithm. $dist[v]$ gives the corresponding distance value of $v$ from $s$. Initially, $unexplored$ contains all node in $g$, and the distance value from $s$ to every node $v \in g$ is $\infty$ except for $s$ itself, whose distance value to $s$ is 0, as shown below:  
\\\\
\tab (initially $unexplored$ contains all nodes in graph $g$)\\
\tab $unexplored = \{v : v \in g\}$
\\\\
\tab (node value is used to index $dist$, initially distance of all nodes are infinity except 
\\ \tab the source node)\\
\tab $dist[s] = 0, dist[a] = \infty, \forall a \in g, a \neq s$
\\\\
We index $unexplored$ and $dist$ by the number of iterations. Specifically, denote $u_i$ as the node being explored at the $i^{th}$ iteration, and denote $dist_i$, $unexplored_i$ as the value of distance list and unexplored list at the beginning of the $i^{th}$ iteration. Then during each iteration the Dijkstra's Algorithm calculates $dist, unexplored, explored$ as follows:
\\
\texttt{
  \tab\tab choose $u_k \in unexplored_k$ s.t. $\forall u' \in unexplored_k, dist_k[u_k] \leq dist_k[u']$ \\
  \tab\tab $unexplored_{k+1} = unexplored_k - \{u_k\}$                    \\
  \tab\tab for($\forall v \in g$) $\{$
  \tab\[
  	dist_{k+1}[v] = min(dist_k[v], (dist_k[u_k] + weight(u_k,v)))
  \]    
  \tab\tab $\}$ \\
}
\\\\

This implementation of Dijkstra's algorithm can be viewed as generating a matrix, where the $i^{th}$ column in the matrix stores the value of $unexplored_i$ and $dist_i$. After calculating a matrix with $n$ columns, the $(n+1)^{th}$ column can be calculated based on the value of $unexplored_n$ and $dist_n$ stored in the last column, i.e., the $n^{th}$ column in the matrix. This representation provides a clear recursive structure for the implementation of Dijkstra's algorithm, and the correctness of the program can be verified by proving that certain properties, for instance distance value of explored nodes stored in each column is the minimum distance value, hold for every column generated. 
