%-----------------------------------------------------------------------------------------------
% Introduction
%-----------------------------------------------------------------------------------------------

Shortest path problems are concerned with finding the path with minimum distance value between two nodes in a given graph. One variation of shortest path problem is single-source shortest path problem, which focuses on finding the path with minimum distance value from one source to all other vertices within the graph. Dijkstra's algorithm \cite{Dijkstras} is one of the most well-known single-source shortest path algorithms, and are implemented in various fields including network protocols and aritificial intelligence.
\\

Given the importance of Dijkstra's in real-life applications, we are interested in verifying the implementation of both algorithms. We first provide concrete implementation for Dijkstra's algorithm, and then define functions with precise type signatures which carry specifications that should hold for the correct implementation, for instance returning the minimum distance value from the source to each node in the graph. Having these functions type checked will then ensure the correctness of our algorithm implementation. Our implementation uses the Idris functional programming language, which embraces powerful tools and features that makes program verification possible. 
\\

Specifically, our contributions are:
\begin{itemize}
	\item Provide a concrete implementation of Dijkstra's algorithm in Idris. 
	\item Offer a verification program for Dijkstra's algorithm written in Idris, which is available on \href{https://github.com/EileenFeng/algorithm_verification}{this} (\url{https://github.com/EileenFeng/algorithm_verification}). Although the proof of some lemmas are incomplete, we are confident that we can provide the complete implementation if granted more time.

\end{itemize}

The structure of this thesis is as follows. Section 2 describes the significance and value of algorithm verification, and reasons of choosing Idris as the language for verifying programs. Section 3 provides some background on Dijkstra's algorithm, follows up by briefly introduction on the Idris functional programming language. Section 4 includes an overview of our verification program, including definition of key concepts, assumptions made by our program, and details on the pseudocode and mathematical proof of Dijkstra's, which serves as important guideline in implementation our verification program. Section 5 covers more details of our verification program, including function type signatures and code of the proof for key lemmas. Section 6 discusses future work. Section 7 presents and compares related work, and section 8 gives a brief conclusion.  