
\section{Related Work}
The increasing importance of Dijkstra's algorithm in many real-world applications has raised an interest on verifying its implementation. Mange and Kuhn provide a project that verifies a Java implementation of Dijkstra's algorithm with the Jahob verification system in their report on efficient proving of Java programs \cite{Mange}. Although the concrete implementation of this work is unavailable, the report demonstrates the verification process. Function behaviors are specified with preconditions, postconditions, and invariants, and Jahob allows programmers to provide these specifications in high-order logic(HOL), which reduces the problem of program verification to the validity of HOL formulas. 
% Existing work on verifying Dijkstra's algorithm is relatively limited, and few resources are found for the verification of Bellman-Ford algorithm. Robin Mange and Jonathan Kuhn demonstrates an implementation that verifies Dijkstra's algorithm with the Jahob verification system in their report on efficient proving of Java implementations[3]. Although few resource has been found on the concrete implementation of this work, the report illustrates that as Jahob allows programmers to provide specification of their function's behaviors in high-level logic(HOL), program verification can be reduced to the problem of the validity of HOL formulas. 
\\

Klasen et. al. verifies Dijkstra's algorithm with the KeY system \cite{Klasen}, an interactive theorem prover for Java. Concrete implementations of Dijkstra's algorithm with different variants are provided, and all of them are written in Java. Similarly to the work by Mange and Kuhn, the verification process in the work by Klasen involves describing the behavior of each function with preconditions, postconditions and modifies clause. Loop invariants are specified to support the verification. A function is then verified as correct by the KeY system, with respect to its behavior specifications, if the postconditions specified hold after execution. A similar implementation is provided by Filliâtre, a senior researcher from the National Center for Scientific Research(CNRS), which verifies Dijkstra's implementation with Why3, a deductive program verification platform that relies on external theorem provers \cite{Jean}\cite{why3}. 
\\

% All works presented above are largely dependent on theorem proving systems, however our work relies on a significantly smaller trusted code base. Most proofs in our work will be implemented from scratches, and considerable amount of details on verification is presented explicitly. This reduces the chance of introducing errors into our verification program due to bugs in the proof management systems, and additionally, provides an example of how program verification can be achieved with a general-purpose programming language, and that the implementation is highly similar to that of any other programs. 



% Jean-Christophe Filliâtre, a senior researcher from the National Center for Scientific Research(CNRS), offers an implementation of Dijkstra's algorithm along with its verification in Why3, a deductive program verification platform that relies on external theorem provers[4][5]. Both verifications above are largely dependent on theorem proving systems. Unlike Filliâtre and Klasen et. al., our work relies on a significantly smaller trusted code base, indicating that considerable amount of proofs will be presented explicitly in our implementation rather than provided by external theorem provers.
