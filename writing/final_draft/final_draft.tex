% Credits are indicated where needed. The general idea is based on a template by Vel (vel@LaTeXTemplates.com) and Frits Wenneker.

\documentclass[11pt, a4paper]{article} % General settings in the beginning (defines the document class of your paper)
% 11pt = is the font size
% A4 is the paper size
% “article” is your document class

%----------------------------------------------------------------------------------------
% Packages
%----------------------------------------------------------------------------------------

% Necessary
\usepackage[german,english]{babel} % English and German language 
\usepackage{booktabs} % Horizontal rules in tables 
% For generating tables, use “LaTeX” online generator (https://www.tablesgenerator.com)
\usepackage{comment} % Necessary to comment several paragraphs at once
\usepackage[utf8]{inputenc} % Required for international characters
\usepackage[T1]{fontenc} % Required for output font encoding for international characters

% Might be helpful
\usepackage{amsmath,amsfonts,amsthm} % Math packages which might be useful for equations
\usepackage{tikz} % For tikz figures (to draw arrow diagrams, see a guide how to use them)
\usepackage{tikz-cd}
\usepackage{hyperref}
\usepackage{import}
\usepackage{listings}

\usetikzlibrary{positioning,arrows} % Adding libraries for arrows
\usetikzlibrary{decorations.pathreplacing} % Adding libraries for decorations and paths
\usepackage{tikzsymbols} % For amazing symbols ;) https://mirror.hmc.edu/ctan/graphics/pgf/contrib/tikzsymbols/tikzsymbols.pdf 
\usepackage{blindtext} % To add some blind text in your paper
\theoremstyle{definition}
\newtheorem{definition}{Definition}[section]

\newcommand\tab[1][1cm]{\hspace*{#1}}
\newcommand\ftab[1][5cm]{\hspace*{#1}}
\newcommand\ttab[1][2cm]{\hspace*{#1}}
\newcommand\tsp[1][0.2cm]{\hspace*{#1}}
\newcommand\htab[1][0.5cm]{\hspace*{#1}}
\newcommand\st{\texttt}
\setcounter{secnumdepth}{4}

\graphicspath{ {images/} }

\newtheorem*{theorem}{Theorem}
\newtheorem*{lemma}{Lemma}
\newtheorem{sublemma}{Lemma}[section]

%---------------------------------------------------------------------------------
% Additional settings
%---------------------------------------------------------------------------------

% adding extra level of sections

%---------------------------------------------------------------------------------
% Define your margins
\usepackage{geometry} % Necessary package for defining margins

\geometry{
  top=2cm, % Defines top margin
  bottom=2cm, % Defines bottom margin
  left=2.5cm, % Defines left margin
  right=2.5cm, % Defines right margin
  includehead, % Includes space for a header
  %includefoot, % Includes space for a footer
  %showframe, % Uncomment if you want to show how it looks on the page 
}

\setlength{\parindent}{15pt} % Adjust to set you indent globally 

%---------------------------------------------------------------------------------
% Define your spacing
\usepackage{setspace} % Required for spacing
% Two options:
\linespread{1.1}
%\onehalfspacing % one-half-spacing linespread

%----------------------------------------------------------------------------------------
% Define your fonts
\usepackage[T1]{fontenc} % Output font encoding for international characters
\usepackage[utf8]{inputenc} % Required for inputting international characters

\usepackage{XCharter} % Use the XCharter font


%---------------------------------------------------------------------------------
% Define your headers and footers

\usepackage{fancyhdr} % Package is needed to define header and footer
\pagestyle{fancy} % Allows you to customize the headers and footers

%\renewcommand{\sectionmark}[1]{\markboth{#1}{}} % Removes the section number from the header when \leftmark is used

% Headers
\lhead{} % Define left header
\chead{\textit{}} % Define center header - e.g. add your paper title
\rhead{} % Define right header

% Footers
\lfoot{} % Define left footer
\cfoot{\footnotesize \thepage} % Define center footer
\rfoot{ } % Define right footer

%---------------------------------------------------------------------------------
% Add information on bibliography
\usepackage{natbib} % Use natbib for citing
\usepackage{har2nat} % Allows to use harvard package with natbib https://mirror.reismil.ch/CTAN/macros/latex/contrib/har2nat/har2nat.pdf

% For citing with natbib, you may want to use this reference sheet: 
% http://merkel.texture.rocks/Latex/natbib.php

%---------------------------------------------------------------------------------
% Add field for signature (Reference: https://tex.stackexchange.com/questions/35942/how-to-create-a-signature-date-page)
\newcommand{\signature}[2][5cm]{%
  \begin{tabular}{@{}p{#1}@{}}
    #2 \\[2\normalbaselineskip] \hrule \\[0pt]
    {\small \textit{Signature}} \\[2\normalbaselineskip] \hrule \\[0pt]
    {\small \textit{Place, Date}}
  \end{tabular}
}
%---------------------------------------------------------------------------------
% General information
%---------------------------------------------------------------------------------
% \title{Shortest Path Algorithms Verification with Idris} % Adds your title
% \author{
% Yazhe Feng % Add your first and last name
%     %\thanks{} % Adds a footnote to your title
%     %\institution{YOUR INSTITUTION} % Adds your institution
%   }

% \date{\small \today} % Adds the current date to your “cover” page; leave empty if you do not want to add a date

%---------------------------------------------------------------------------------
% Define what’s in your document
%---------------------------------------------------------------------------------
\usepackage{xcolor}% http://ctan.org/pkg/xcolor
\definecolor{gray_ulisses}{gray}{0.7}
\definecolor{castanho_ulisses}{rgb}{0.71,0.33,0.14}
\definecolor{preto_ulisses}{rgb}{0.41,0.20,0.04}
\definecolor{green_ulises}{rgb}{0.3,0.75,0.1}
\definecolor{dark_yellow}{RGB}{247, 191, 9}
\definecolor{kw_pink}{RGB}{155, 82, 121}
\definecolor{kw_green}{RGB}{0,102,51}
\definecolor{vgreen}{RGB}{123, 135, 21}
\definecolor{kw_ground}{RGB}{204,102,0}
\definecolor{kw_orange}{RGB}{255, 128, 0}
\definecolor{kw_blue}{RGB}{0,102,204}
\definecolor{kw_red}{RGB}{255, 51, 51} %{211, 78, 63}

\lstdefinelanguage{Idris} {
  %morekeywords={->,=,:},
  basicstyle=\fontsize{10}{11}\selectfont\ttfamily,
  sensitive=true,
  morecomment=[l][\color{blue}]{--},
  morecomment=[s][\color{blue}]{\{-}{-\}},
  morestring=[b]",
  stringstyle=\color{red},
  showstringspaces=false,
  numberstyle=\tiny,
  numberblanklines=true,
  showspaces=false,
  breaklines=true,
  showtabs=false,
  emph=
  {[1]
    ->,=,:
  },
  emphstyle={[1]\color{kw_red}\textbf},
  emph=
  {[2]
    Bool,Char,List,Vect,Nat,Type,Int,Maybe,String,Fin
  },
  emphstyle={[2]\color{kw_orange}\textbf},
  emph=
  {[3]
    case,of,proof,with,data,record,where,import,in,module,if,then,else,total,\%default,?
  },
  emphstyle={[3]\color{kw_red}\textbf},
  % emph=
  % {[4]
  %   quot,rem,div,mod,elem,notElem,seq
  % },
  % emphstyle={[4]\color{castanho_ulisses}\textbf},
  emph=
  {[4]
    True,False,Just,Nothing,Refl,LTE,FZ,FS,Z,S
  },
  emphstyle={[4]\color{kw_green}\textbf}
  % emph=
  % {[5]
  %   Graph, Column, Node, nodeset, WeightOps
  % },
  % emphstyle={[5]\color{kw_blue}\textbf}
}
\newcommand\inl{\lstinline}




\begin{document}
\lstset{
  language=Idris, 
  morekeywords={->,=,:},
  backgroundcolor=\color{white},
  keywordstyle=\color{kw_orange},
  showtabs=false,     
  showspaces=false,                
  showstringspaces=false,             
  tabsize=2  
}

% \lstset{
%   language=Haskell, 
%   backgroundcolor=\color{white}
%   frame=single,                    
%   keepspaces=true,                 
%   keywordstyle=\color{blue},
%   basicstyle=\fontsize{10}{12}\selectfont\ttfamily,
%   breaklines=true, 
%   commentstyle=\color{orange}, 
%   showtabs=false,     
%   showspaces=false,                
%   showstringspaces=false,             
%   tabsize=2
% }


\begin{titlepage}
\pagenumbering{gobble}
\addcontentsline{toc}{section}{Abstract}
\begin{center}
\tab\\
\vspace{1.5cm}
\LARGE
\text{\uppercase{Verification of Dijkstra's Algorithm}}\\
\text{\uppercase{ in Idris}}

\Large
\vspace{1.5cm}
\text{Yazhe Feng}\\
\vspace{2cm}
\large
A thesis submitted in partial fulfillment of the requirements for the \\ degree of Bachelor of Arts \\ 
\vspace{0.5cm}
\large
in the \\ 
\vspace{0.5cm}
\large
Department of Computer Science \\ 
\vspace{0.5cm}
\large
at Bryn Mawr College \\
\vspace{1cm}
Advisor: Richard A. Eisenberg

\pagebreak

\end{center}
\pagenumbering{roman}
\Large
\textbf{Acknowledgments} 
\\\\
\small
%\addcontentsline{toc}{section}{Acknowledgments}
(To be finished...)

\pagebreak
\begin{center}
\large
\textbf{Abstract}
\end{center}
\pagebreak
%\end{center}

\end{titlepage}

\vfill
% \begin{figure}[H]
% \begin{center}
% \includegraphics[width = .2\textwidth]{university_seal}
% \end{center}
% \end{figure}
% You can add your university seal here. It'll appear right above the department and it'll look fancy af.
%\end{titlepage}
\pagebreak
\pagenumbering{roman}
\tableofcontents
%\listoffigures
%\listoftables
\pagebreak
\pagenumbering{arabic}

% If you want a cover page, uncomment "%---------------------------------------------------------------------------------
% Cover page
%---------------------------------------------------------------------------------

% Here are more templates for other cover pages: https://www.latextemplates.com/cat/title-pages

% This example is based on this cover page example: https://www.latextemplates.com/template/academic-title-page

\begin{titlepage} % Starts new environment where the page number is not displayed and the count starts at 1 for the next page

%------------------------------------------------
%	Institutional information
%------------------------------------------------
	
\begin{minipage}{0.4\textwidth} % Begins new environment (like a text box)
    \begin{flushleft} % Sets environment on the left side of the paper
    \large
    University of XX\\ % Add your institution
    Chair of Political Science IV\\ % Add the chair
    Fall 2018\\ % Add term
    COURSE TITLE\\ % Add course title
    Supervisor: NAME % Add instructor/supervisor name 
    \end{flushleft}
\end{minipage}
	
\vspace*{2in} % Adds some space in-between
	
\center % Centre everything on the page

%------------------------------------------------
%	Main part
%------------------------------------------------
	
{\huge\bfseries TITLE OF YOUR PAPER}\\[0.4cm] % Add your paper title 
{\large\today}\\[0.4cm] % Add date (current day)
FIRSTNAME LASTNAME % Add your name
	
\vfill % Adds additional space

%------------------------------------------------
%	General information about the author
%------------------------------------------------

\vfill % Adds additional space

Your contact info \\ % Add your contact info
Your Program \\ % Add info about your program
Semester you are enrolled \\ % Add info about your semester

\vfill % Adds additional space

%------------------------------------------------
%	Word count
%------------------------------------------------

\vfill % Adds additional space
	
Word count: XXXX % To indicate the word count
% How to check words in a LaTeX document: https://www.overleaf.com/help/85-is-there-a-way-to-run-a-word-count-that-doesnt-include-latex-commands
	

	
\end{titlepage}" and uncomment "\begin{comment}" and "\end{comment}" to comment the following lines
%%---------------------------------------------------------------------------------
% Cover page
%---------------------------------------------------------------------------------

% Here are more templates for other cover pages: https://www.latextemplates.com/cat/title-pages

% This example is based on this cover page example: https://www.latextemplates.com/template/academic-title-page

\begin{titlepage} % Starts new environment where the page number is not displayed and the count starts at 1 for the next page

%------------------------------------------------
%	Institutional information
%------------------------------------------------
	
\begin{minipage}{0.4\textwidth} % Begins new environment (like a text box)
    \begin{flushleft} % Sets environment on the left side of the paper
    \large
    University of XX\\ % Add your institution
    Chair of Political Science IV\\ % Add the chair
    Fall 2018\\ % Add term
    COURSE TITLE\\ % Add course title
    Supervisor: NAME % Add instructor/supervisor name 
    \end{flushleft}
\end{minipage}
	
\vspace*{2in} % Adds some space in-between
	
\center % Centre everything on the page

%------------------------------------------------
%	Main part
%------------------------------------------------
	
{\huge\bfseries TITLE OF YOUR PAPER}\\[0.4cm] % Add your paper title 
{\large\today}\\[0.4cm] % Add date (current day)
FIRSTNAME LASTNAME % Add your name
	
\vfill % Adds additional space

%------------------------------------------------
%	General information about the author
%------------------------------------------------

\vfill % Adds additional space

Your contact info \\ % Add your contact info
Your Program \\ % Add info about your program
Semester you are enrolled \\ % Add info about your semester

\vfill % Adds additional space

%------------------------------------------------
%	Word count
%------------------------------------------------

\vfill % Adds additional space
	
Word count: XXXX % To indicate the word count
% How to check words in a LaTeX document: https://www.overleaf.com/help/85-is-there-a-way-to-run-a-word-count-that-doesnt-include-latex-commands
	

	
\end{titlepage}

%\begin{comment}
%\maketitle % Print your title, author name and date; comment if you want a cover page 

% \begin{center} % Center text
%     Word count: XXXX
% % How to check words in a LaTeX document: https://www.overleaf.com/help/85-is-there-a-way-to-run-a-word-count-that-doesnt-include-latex-commands
% \end{center}
% %\end{comment}

%----------------------------------------------------------------------------------------
% Introduction
%----------------------------------------------------------------------------------------
\setcounter{page}{1} % Sets counter of page to 1

\section{Introduction}
\import{./}{introduction}

% \subsection{Citing} % Add another subsection
% Citing in \LaTeX is easy. You could easier cite with the text flow like this ``Referring to \citet{collier2004greed} ...''  or at the end of the sentence \cite{collier2004greed}. You can also cite pages like this \citep[55]{collier2004greed}. If you want to add an additional note, you might want to do it this way \citep[cp.][22]{collier2004greed} or like this \citep[cp.][]{collier2004greed}.\\
% \blindtext % Adds some blintext to your text

%----------------------------------------------------------------------------------------
% Literature review
%----------------------------------------------------------------------------------------
\section{Motivation}
\import{./}{motivation}

\section{Background}
\import{./}{background}

\section{Overview of Dijkstra's Implementation and Proof of Correctness}
This section provides an overview of our Dijkstra's implementation and mathematical proof of correctness. Section \ref{definitions} provides definitions of key concepts used in our work, and Section \ref{lemmas} presents the lemmas of our proof. The lemma statements and the proofs on Lemma \ref{lemma4.1} and Lemma \ref{lemma4.3} are 

\import{dijkstras_highlevel/}{data_structure}

\import{dijkstras_highlevel/}{definitions}

\import{dijkstras_highlevel/}{pseudocode}

\import{dijkstras_highlevel/}{correctness_proof}

%---------------------------------------------------------------------------------
% Low-Level Contribution // need to change the title name
%---------------------------------------------------------------------------------

\section{Concrete Implementation of Dijkstra's Verification}
\import{verification/}{data_structures.tex}
\import{verification/}{implementation.tex}
\import{verification/}{verification.tex}


%----------------------------------------------------------------------------------------
% Discussion
%----------------------------------------------------------------------------------------

\section{Discussion}



%----------------------------------------------------------------------------------------
% Related Work
%----------------------------------------------------------------------------------------
\import{./}{related_work}


%----------------------------------------------------------------------------------------
% Conclusion
%----------------------------------------------------------------------------------------

\section{Conclusion}

%----------------------------------------------------------------------------------------
% Bibliography
%----------------------------------------------------------------------------------------
\newpage % Includes a new page

\pagenumbering{roman} % Changes page numbering to roman page numbers
%\bibliography{literature}
\medskip
\bibliographystyle{unsrt} % Defines your bibliography style
\bibliography{literature} % Add the filename of your bibliography
% [1] V. Klasen, "Verifying dijkstra's algorithm with key," in Diploma Thesis, Universitat Koblenz-Landau, 2010.%\url{https://kola.opus.hbz-nrw.de/opus45-kola/frontdoor/deliver/index/docId/420/file/DA_KLASEN.pdf}
% \newline
% [2] Bove, Ana, and Peter Dybjer. “Dependent Types at Work.” Language Engineering and Rigorous Software Development Lecture Notes in Computer Science, 2009, pp. 57–99., \url{doi:10.1007/978-3-642-03153-3_2}. %\url{http://www.cse.chalmers.se/~peterd/papers/DependentTypesAtWork.pdf}
% \newline
% [3] R. Mange and J. Kuhn, "Verifying dijkstra algorithm in Jahob," 2007, student project, EPFL.%\url{https://lara.epfl.ch/w/_media/dijkstra.pdf} 
% \newline
% [4] Filliâtre, Jean-Christophe. “Toccata.” Dijkstra's Shortest Path Algorithm, \url{toccata.lri.fr/gallery/dijkstra.en.html}.%\url{http://toccata.lri.fr/gallery/dijkstra.en.html}
% \newline
% [5] “Why3.” Why3, May 2015, \url{why3.lri.fr/#provers}. %\url{http://why3.lri.fr/#provers}
% \newline
% %[6] \url{http://lara.epfl.ch/w/jahob_system}
% [6] E. W. Dijkstra. A note on two problems in connexion with graphs. Numerische Mathematik, 1:269–271, 1959.
% \newline 
% [7] Richard E. Bellman. On a routing problem. Quarterly of Applied Mathematics, 16:87–90, 1958.
% \newline
% [8] A. Shimbel. Structure in communication nets. Polytechnic Press of the Poly- technic Institute of Brooklyn, page 199–203, 1955.
% \newline
% [9] Dijkstra (1970) \href{http://www.cs.utexas.edu/users/EWD/ewd02xx/EWD249.PDF}{"Notes On Structured Programming"} (EWD249), Section 3 ("On The Reliability of Mechanisms"), corollary at the end.
% For citing, please see this sheet: http://merkel.texture.rocks/Latex/natbib.php

%----------------------------------------------------------------------------------------
% Appendix
%----------------------------------------------------------------------------------------
\newpage % Includes a new page
\section*{Appendix} % Stars disable section numbers
% \appendix % Uncomment if you want to add an "automatic" appendix
\pagenumbering{Roman} % Changes page numbering to Roman page numbers


%----------------------------------------------------------------------------------------
% Declaration
%----------------------------------------------------------------------------------------
\newpage % Includes a page break
\thispagestyle{empty} % Leaves the page style empty (no page number, no header, no footer)
\section*{Statutory Declaration} % Stars disable section numbers

\vspace*{1in} % Adds extra space between two paragraphs


%\vspace*{1in} % Adds extra space
% % Add field for signature, date, and place
% \hfill \signature{} 


%---------------------------------------------------------------------------------

\end{document}
