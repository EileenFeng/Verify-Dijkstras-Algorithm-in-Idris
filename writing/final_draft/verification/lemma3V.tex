\paragraph{Lemma 3 - \inl{l3_preserveDelta}} \label{lemma3V}
\tab\\\\
In verifying Dijkstra's correctness, it is important to show that forall nodes \inl{v} in the input graph, once the distance value calculated for \inl{v} is equal to the minimum distance value from the source node to \inl{v}, then the distance value of \inl{v} does not change through the execution of the algorithm. This is proved by Lemma \ref{4.3} in the mathematical proof of Dijkstra's algorithm, and implemented by the function \inl{l3_preserveDelta} below (the proof of \inl{l3_preserveDelta} is provided in later paragraphs). 
\begin{lstlisting}
	l3_preserveDelta : {g : Graph gsize weight ops} ->
	                   (cl : Column (S len) g src) ->
	                   (l2_ih : neDInfPath cl) ->
	                   (v : Node gsize) ->
	                   (psv : Path src v g) ->
	                   (psv_sv : shortestPath g psv) ->
	                   (eq : dEq ops (nodeDistN v cl) (length psv) = True) ->
	                   dEq ops (nodeDistN v (runHelper cl)) (length psv) = True
\end{lstlisting}

\inl{l3_preserveDelta} states that given a \inl{Column} named \inl{cl}, for any node \inl{v} in graph, given a shortest path named \inl{psv} from \inl{src} to \inl{v} (specified by \inl{psv_sv}), if the distance value of \inl{v} stored in \inl{cl} is equal to the length of \inl{psv}(stated by the input \inl{eq : dEq ops (nodeDistN v cl) (length psv) = True}), then the distance value of \inl{v} stored in \inl{runHelper cl} is also equal to the length of \inl{psv}. Since the proof of Lemma 4.3 is based on Lemma 4.2 as we mentioned at the beginning of Section \ref{lemmas}, the proof of property \inl{neDInPath} on \inl{cl} is provided by the input \inl{l2_ih : neDInfPath cl}. 
\\

We prove \inl{l3_preserveDelta} by showing that the distance of a node \inl{v} stored in a \inl{Column} is non-increasing (either decrease or remain the same) each time after calling \inl{runHelper}. If the current distance stored in \inl{Column} for \inl{v} is equal to the minimum distance value, than there cannot exist a smaller distance value for \inl{v}, hence the distance stored for \inl{v} remains unchanged each time after calling \inl{runHelper}.
\\

To implement \inl{l3_preserveDelta}, we first need to show that the distance value stored for all node is non-increasing after each call of \inl{runHelper}. The function \inl{runDecre} provided below states this property. We provide a detailed discussion on the implementation of \inl{runDecre} as it presents how we approach the proofs of some key lemmas in our verification. Specifically, we break the statement that we want to prove into smaller ones by destructing the data structures involved in the original statement. In other words, the implementation of functions that involve more complex data types can be built on functions that deal with simpler data types, which are easier to approach. The following explanation on the implementation of \inl{runDecre} illustrates this technique. 
\begin{lstlisting}
	runDecre : {g : Graph gsize weight ops} ->
	           (cl : Column (S len) g src) ->
	           (v : Node gsize) ->
	           dgte ops (nodeDistN v cl) 
	           			(nodeDistN v (runHelper cl)) = True
	runDecre (MKColumn g src (S len) unexp dist) (MKNode nv) {gsize} {ops} {weight}
	  = distDecre g min_node min_dist (mkNodes gsize) dist 
	  			  (finToNat nv) {p=nvLTE nv}
	  where
	   ...
\end{lstlisting}

The return type of the \inl{runDecre} function specifies that for all node \inl{v}, the distance value stored for \inl{v} in \inl{cl} is either decreasing, or maintains the same after each call of \inl{runHelper} on \inl{cl}. Since \inl{runDecre} involves the \inl{Column} data type, and the main field in \inl{Column} that concerns us here is the \inl{Vect} of distance values, the implementation of \inl{runDecre} is built on a function \inl{distDecre}, which states the same non-increasing property of distance values calculated but only involves the \inl{Vect} of distance values instead. 
As the function \inl{runHelper} calls \inl{updateDist} for updating the distance value \inl{Vect} (mentioned in Section \ref{second_layer}), the type of \inl{distDecre} also involves the \inl{updateDist} function. Since we use the \inl{Fin} type value carried by node \inl{v} to index its distance value stored in \inl{cl} (Section \ref{column}), and recursing on a \inl{Fin} type value with base case \inl{FZ} introduces more complications, we define a \inl{finToNat} function that convert a \inl{Fin} into its corresponding \inl{Nat} value. As in the definition of \inl{runDecre}, the \inl{distDecre} function is called on the corresponding \inl{Nat} value of \inl{nv} calculated by the function \inl{finToNat}. The implementation of \inl{distDecre} is presented below.
\begin{lstlisting}
	distDecre : (g : Graph gsize weight ops) ->
	            (mn : Node gsize) ->
	            (min_dist : Distance weight) ->
	            (nodes : Vect m (Node gsize)) ->
	            (dist : Vect m (Distance weight)) ->
	            (nv : Nat) ->
	            {auto p : LT nv m} ->
	            dgte ops (indexN nv dist) 
	            		 (indexN nv (updateDist g mn min_dist nodes dist)) 
	            		 = True
	distDecre g mn _ Nil Nil nv {p} = absurd $ succNotLTEzero p
	distDecre g mn min_dist (n :: ns) (d :: ds) Z 
		= calcDistEq g mn n min_dist d
	distDecre g mn min_dist (n :: ns) (d :: ds) (S nv) {p=LTESucc p'} 
		= distDecre g mn min_dist ns ds nv {p= p'}
\end{lstlisting}

Inputs of \inl{distDecre} include the current node with minimum distance, named as \inl{mn}, and its distance value \inl{min_dist}; \inl{nodes} and \inl{dist} denotes a \inl{Vect} of nodes in the graph and a \inl{Vect} of the corresponding distance values for these nodes respectively. As \inl{distDecre} recurs on both \inl{nodes} and \inl{dist}, both \inl{Vect}s have the same length \inl{m} but the value of \inl{m} changes during each recursive step. 
\\

The \inl{nv} argument is the index of the node \inl{v}(from the \inl{runDecre} function) in \inl{nodes} and its corresponding distance value in \inl{dist}, with respect to the current recursive step. Specifically, as we define the \inl{distDecre} function to recur on \inl{nodes}, \inl{dist}, and \inl{nv} simultaneously, the length of \inl{nodes} and \inl{dist}, and the value of \inl{nv} decreases by one during each recursive step until one of them reaches \inl{Z}, which is the base case. For instance if the the value of \inl{nv} is `S Z' during the current recursion, then in the next recursion the value passed in for \inl{nv} should `Z', however as \inl{distDecre} recurs on the tails of \inl{nodes} and \inl{dist}, \inl{nv} denotes the locations of the same node in \inl{nodes} and its corresponding distance value in \inl{dist} during every recursive step. Lastly, the implicit parameter \inl{p} is a proof stating that the current index \inl{nv} is smaller than the length of \inl{Vect}s \inl{m}, which ensures safe indexing. The return type of \inl{distDecre} specifies that the distance of node indexed by \inl{nv} stored in \inl{dist} is non-decreasing after calling \inl{updateDist} on \inl{dist}. 
\\

Similar to the structure of \inl{runDecre}, the \inl{distDecre} function is again built on a \inl{calcDistEq} function, which concerns the distance value for one specific node rather than a \inl{Vect} of distance values. The definition of \inl{calcDistEq} is provided below. 
\begin{lstlisting}
		calcDistEq : (g : Graph gsize weight ops) ->
		             (min_node : Node gsize) ->
		             (cur : Node gsize) ->
		             (min_dist : Distance weight) ->
		             (cur_dist : Distance weight) ->
		             dgte ops cur_dist 
		             		  (calcDist g min_node cur min_dist cur_dist) = True
		calcDistEq g min_node cur min_dist cur_dist {ops}
		  with (dgte ops cur_dist 
		  				 (dplus ops (edgeW g min_node cur) min_dist)) proof sdist
		    | True = sym sdist
		    | False = dgteRefl
\end{lstlisting}

The return type of \inl{calcDistEq} states that the distance value for the node \inl{cur} (named as \inl{cur_dist}) is smaller or equal to that after running \inl{calcDist} on \inl{cur}. Based on the implementation of \inl{calcDist} in Section \ref{third_layer}, the proof of \inl{calcDistEq} is directly given by matching on the result of checking whether \inl{cur_dist} is greater than or equal to the sum of \inl{min_dist} and edge weight between \inl{min_node} and \inl{cur} using the \inl{with} rule. When \inl{cur_dist} is greater(or equals), then the proof \inl{sdist} generated by the match states exactly what we want to prove. When \inl{cur_dist} is smaller than the sum, then the distance value of \inl{cur} calculated by \inl{calcDist} should be \inl{cur_dist} as \inl{calcDist} chooses the minimum distance value between the two, hence we simply call the \inl{dgteRefl} function, which states that a distance value is greater than or equal to itself. 
\\

Based on the \inl{runDecre} function defined above, we offer the following implementation for \inl{l3_preserveDelta}. 
\begin{lstlisting}
	l3_preserveDelta : {g : Graph gsize weight ops} ->
	                   (cl : Column (S len) g src) ->
	                   (l2_ih : neDInfPath cl) ->
	                   (v : Node gsize) ->
	                   (psv : Path src v g) ->
	                   (psv_sp : shortestPath g psv) ->
	                   (eq : dEq ops (nodeDistN v cl) (length psv) = True) ->
	                   dEq ops (nodeDistN v (runHelper cl)) (length psv) = True
	l3_preserveDelta cl l2_ih v psv psv_sp eq {g} {ops} {src}
	  with (l2_existPath cl l2_ih v 
	  			(dgteDInfTrans {ops=ops} 
	  				(nodeDistN v cl) 
	  				(nodeDistN v (runHelper cl)) 
	  				(pathlenNotDInf (nodeDistN v cl) psv eq) 
	  				(runDecre cl v)))
	    | (lpath ** runclv_lp) = dgteEq (dgteEqTrans runclv_lp True (psv_sp lpath)) 
	    								(dgteEqTrans (dEqComm eq) True (runDecre cl v))
\end{lstlisting}

Given that \inl{neDInfPath} holds for \inl{cl} (specified by input argument \inl{l2_ih}), we first apply Lemma 4.2 \inl{l2_existPath} to show that the distance value of \inl{v} after running \inl{runHelper}, i.e., \inl{`nodeDistN v (runHelper cl)'} is the length of some \inl{src-to-v} path. In order to apply \inl{l2_existPath}, we need to show that the value of \inl{`nodeDistN v (runHelper cl)'} is smaller than \inl{DInf} (infinity distance value). \inl{dgteDInfTrans} is a proof stating that, given two distance values \inl{d1, d2}, if \inl{d1} is smaller than \inl{DInf}, and \inl{d2} is smaller than \inl{d1}, then \inl{d2} is also smaller than \inl{DInf}. When applying \inl{dgteDInfTrans} in the implementation of \inl{l3_preserveDelta} above, \inl{`nodeDistN v cl'} corresponds to \inl{d1} and \inl{`nodeDistN v (runHelper cl)'} corresponds to \inl{d2}. The \inl{pathlenNotDInf} function states that the length of any path is smaller than \inl{DInf}. Since we know \inl{`nodeDistN v cl'} is the length of a path \inl{psv} (specified by input argument \inl{eq}), then \inl{`nodeDistN v cl'} is smaller than \inl{DInf}. \inl{`runDecre cl v'} states that the value of \inl{`nodeDistN v (runHelper cl)'} is smaller than or equal to \inl{`nodeDistN v cl'}. By applying \inl{dgteDInfTrans} to all of these information above, we obtained the dependent pair \inl{(lpath ** runclv_lp)}, where \inl{lpath} is a \inl{src-to-v} path, and \inl{runclv_lp} is a proof that the value of \inl{`nodeDistN v (runHelper cl)'} is the length of \inl{lpath}. 
\\

The \inl{dgteEq} function states that for two distance values \inl{d1, d2}, if \inl{d1} is greater than or equal to \inl{d2} and \inl{d2} is greater than or equal to \inl{d1}, then \inl{d1} must equal to \inl{d2}. The \inl{dgteEqTrans} function states that, given three distance values \inl{d1, d2, d3}, if \inl{d1} equals to \inl{d2}, and \inl{d2} is smaller(or greater) than \inl{d3}, then \inl{d1} is smaller(or greater) than \inl{d3}. By applying \inl{dgteEqTrans} to \inl{runclv_lp} and \inl{`psv_sp lpath'} (which states that the length of \inl{lpath} is smaller than the length of \inl{psv}), we know that the value of \inl{`nodeDistN v (runHelper cl)'} is greater than or equal to \inl{`length psv'}(mark as [S1]). However applying \inl{dgteEqTrans} to \inl{eq} and {`runDecre cl v'} we obtained that the \inl{`nodeDistN v (runHelper cl)'} should be smaller than the length of \inl{psv}(mark as [S2]). Hence by applying \inl{dgteEq} to [S1] and [S2], we know that \inl{`nodeDistN v (runHelper cl)'} is equal to the length of \inl{psv}. The implementation of \inl{l3_preserveDelta} proof is complete. 
\\
