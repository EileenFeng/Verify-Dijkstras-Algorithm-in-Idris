\subsection{Verification of Dijkstra's Algorithm} \label{verification}
Our verification of Dijkstra's algorithm is based on and has a similar structure as the implementation in section \ref{implementation}. Instead of proving Dijkstra's correctness based on the \inl{dijkstras} and \inl{runDijkstras} functions directly, we approach the verification by proving that certain properties maintain for each new \inl{Column} value generated by every call to the \inl{runHelper} function. Specifically, since the \inl{Column} structure carries information on the unexplored nodes and distance values calculated for all nodes in the graph, we can re-state Lemma \ref{lemma4.2} to Lemma \ref{lemma4.5} in the mathematical proof of Dijkstra's correctness(in Section \ref{mproof}) as properties on \inl{Column}, and prove that if these properties preserve after calling \inl{runHelper}. As \inl{runHelper} is called by the \inl{runDijkstras} function, the implementation of our verification follows the same structure by defining a function that recursively applies the above proof of properties preservation, and shows that same properties also hold for the last \inl{Column} value calculated, i.e., the value returned by the \inl{runDijkstras} function, which verifies the correctness of Dijkstra's algorithm. 
\\

In the following sections, we first provide proofs of lemmas that state the properties preservation of each new \inl{Column} generated by \inl{runHelper}, and then present the functions that directly verifies the correctness of Dijkstra's algorithm. As the implementation of all proofs are highly complicated and involves significantly amount of details, the following only elaborates on the implementation of two lemma proofs for the purpose of presenting some techniques on how proofs are approached in our verification, and discuss on the types of other lemmas instead. As this thesis aims to verify Dijkstra's algorithm with the Idris type checker, the types of proofs should provide sufficient information on our verification program.
\\

\import{verification/}{lemmas.tex}
\import{verification/}{correctness.tex}


