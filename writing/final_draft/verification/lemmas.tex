\subsubsection{Lemmas} \label{lemmas}
The mathematical proof of Dijkstra's algorithm in section \ref{mproof} includes five lemmas, however in implementing our verification program, we found it easier to approach by merging Lemma \ref{lemma4.4} into Lemma \ref{lemma4.5} as one of its statements. Lemma 4.1 to 4.5 are defined in order, meaning that the proof of Lemma 4.2 is built on Lemma 4.1, and proof of Lemma 4.3 is built on Lemma 4.1 and 4.2 etc. The implementation of Lemma 4.5 (function \inl{l5_spath}) is directly applied in verifying Dijkstra's correctness, and implementations of Lemma 4.1 to Lemma 4.4 are helper proofs for proving Lemma 4.5
\\

The following first presents the types for all lemmas of our verification program, and then elaborate on the implementation of one of the lemma proofs, in order to provide more insights into how we approach proofs generally. We choose to expand on the proofs of Lemma \ref{lemma4.1} (which corresponds to function \inl{l1_prefixSP}) and Lemma \ref{lemma4.3} (which corresponds to the \inl{l3_preserveDelta} function), as compare to other lemma proofs, proof of Lemma 4.3 involves less details but presents enough information on our techniques in implementing proofs. 
\\

\import{./}{lemma1V}

The structure of the implementation of Lemma 4.2 and Lemma 4.5 is as follows: we first define functions that specifies the \inl{Column} properties stated by each lemma, and then implement a function that proves the preservation of these properties. This structure provides a more clear and straightforward type signatures for our functions in the verification program by separating the types that specifies \inl{Column} properties from the types of the proofs. 
\\

\import{./}{lemma2V}
\import{./}{lemma3V}
\import{./}{lemma5V}