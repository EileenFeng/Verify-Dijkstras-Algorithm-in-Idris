\paragraph{Lemma 2 - \inl{l2_existPath}} \label{lemma2V}
\tab\\\\
In our mathematical proof of Dijkstra's correctness, Lemma \ref{lemma4.2} states that given an input graph \inl{g}, for all nodes \inl{v} in \inl{g}, if $dist_{n+1}[v] \neq \infty$, then $dist_{n+1}[v]$ is the length of some \inl{s-to-v} path in \inl{g}. As mentioned at the beginning of this section, in our verification program, we state Lemma 4.2 as a \inl{Column} property and prove these properties preserve after calling \inl{runHelper}. The function \inl{neDInfPath} provided below specifies the \inl{Column} property stated by Lemma \ref{lemma4.2}.
\begin{lstlisting}
		neDInfPath : {g : Graph gsize weight ops} ->
		             (cl : Column len g src) ->
		             Type
		neDInfPath cl {g} {src} {ops} {gsize}
		  = (v : Node gsize) ->
		    (ne : dgte ops (nodeDistN v cl) DInf = False) ->
		    (psv : Path src v g ** dEq ops (nodeDistN v cl) (length psv) = True)
\end{lstlisting}
 
Given \inl{cl} with type \inl{Column len g src}, the function \inl{neDInfPath} specifies that for any node \inl{v} in the graph, if the distance value of \inl{v} stored in \inl{cl} is smaller than infinity, then it is the length of some path from \inl{src} to \inl{v} in \inl{g}. \inl{nodeDistN} is a function that indexes the distance value for a specific node in a \inl{Column}, and in the definition of \inl{neDInfPath}, \inl{nodeDistN v cl} gets the distance value of \inl{v} stored in \inl{cl}, and the dependent pair \inl{(psv : Path src v g ** dEq ops (nodeDistN v cl) (length psv) = True)} specifies the existence of a path \inl{psv} from \inl{src} to \inl{v} in \inl{g}, such that the distance value of \inl{v} stored in \inl{cl} is the length of \inl{psv}. We then define the type of the function \inl{l2_existPath} that states the preservation of the \inl{neDInfPath} property. 
\begin{lstlisting}
		l2_existPath : {g : Graph gsize weight ops} ->
		               (cl : Column (S len) g src) ->
		               (l2_ih : neDInfPath cl) ->
		               neDInfPath (runHelper cl)
\end{lstlisting}

\inl{l2_existPath} states that given \inl{cl} with type \inl{Column (S len) g src}, if \inl{neDInfPath} holds for \inl{cl} (specified by \inl{l2_ih}), then it also holds for the column generated by \inl{(runHelper cl)}. Notice that the input \inl{cl} of \inl{l2_existPath} contains a non-empty vector of unexplored nodes, which is restricted by the \inl{runHelper} function. Similar to the previous proof on \inl{l1_prefixSP}, we can bring the node \inl{v} and statement \inl{ne : dgte ops (nodeDistN v cl) DInf = False} in \inl{neDInfPath} into scope. The proof of \inl{l2_existPath} is approached by matching on the distance value of \inl{v} stored in the \inl{Column} generated by \inl{runHelper cl}. If the distance value of \inl{v} in \inl{runHelper cl} is the same as that in \inl{cl}, then the proof is given by \inl{l2_ih}. Otherwise we check whether \inl{v} is equal to \inl{getMin cl} (the unexplored node with minimum distance value chosed by the algorithm for exploring its neighbors, mentioned in Section \ref{second_layer}), and prove both cases by applying \inl{l2_ih} on \inl{(getMin cl)}. In the case when \inl{v} is not equal to \inl{getMin cl}, the proof is still incomplete due to an unclear type error introduced by calling the \inl{inNodeset} function, as discussed under Section \ref{discussion}.
\\
