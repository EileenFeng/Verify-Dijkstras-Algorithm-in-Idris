\subsubsection{Verification of Correctness}
The lemma proofs implementation discussed in the previous section provides us a function \inl{l5_spath} which states that if the \inl{Column} properties specified by the functions \inl{neDInfPath} (in Section \ref{lemma2V}) and \inl{l5_stms}(in Section \ref{lemma5V}) hold for a \inl{Column} \inl{cl}, then the properties perserve after calling \inl{runHelper} on \inl{cl}. Based on \inl{l5_spath} we define the below function \inl{correctness}, which states that if the first \inl{Column} named \inl{cl} generated in the \inl{dijkstras} function in our Dijkstra's implementation (Section \ref{first_layer}), fulfills the properties stated by \inl{neDInfPath} and \inl{l5_spath}, then the properties reserve after callign \inl{runDijkstras} on \inl{cl}. The function \inl{correctness} is defined by recursively applying the lemmas \inl{l2_existPath} and \inl{l5_spath}, as presented below. 
\\
\begin{lstlisting}
correctness : {g : Graph gsize weight ops} ->
              (cl : Column len g src) ->
              (nadj : ((n : Node gsize) -> 
                            inNodeset n (getNeighbors g n) = False)) ->
              (l2_ih : neDInfPath cl) ->
              (l5_ih : l5_stms cl) ->
              l5_stms (runDijkstras cl)
correctness {len = Z} cl nadj l2_ih l5_ih = l5_ih
correctness {len=S n} cl@(MKColumn g src (S n) unexp dist) nadj l2_ih l5_ih
  = correctness (runHelper {len=n} cl) nadj 
  				      (l2_existPath cl l2_ih) 
  				      (l5_spath cl nadj l2_ih l5_ih)
\end{lstlisting}


\inl{nadj} is a function that states a node cannot be in the \inl{nodeset} of itself, which is required by the function \inl{l5_spath}. \inl{l2_ih} and \inl{l5_ih} states that the properties specified by function \inl{neDInfPath} and \inl{l5_stms} hold for the input \inl{Column} \inl{cl}, and the return type states that \inl{l5_stms} also holds for the \inl{Column} generated by running \inl{runDijkstras} on \inl{cl}. \inl{correctness} is defined recursively. When the input \inl{cl} contains no unexplored nodes, then the input argument \inl{l5_ih} directly provides the definition. Otherwise, since \inl{runDijkstras} calls \inl{runHelper} for updating the \inl{Column} value, we recur the \inl{correctness} function on the new \inl{Column} generated by \inl{`runHelper cl'}, and updates the corresponding inputs by applying lemmas \inl{l2_existPath} and \inl{l5_spath} on \inl{cl}. 
\\

We then defined the \texttt{dijkstras\_correctness} function presented below, which wraps up all lemmas and helper proof, and verify the minimum distance property for all nodes in the input graph. 
\\
\begin{lstlisting}
dijkstras_correctness : (gsize : Nat) ->
                        (g : Graph gsize weight ops) ->
                        (src : Node gsize) ->
                        (v : Node gsize) ->
                        (psv : Path src v g) ->
                        (spsv : shortestPath g psv) ->
                        (nadj : ((n : Node gsize) -> 
                        		inNodeset n (getNeighbors g n) = False)) ->
                        dEq ops (indexN (finToNat (getVal v)) 
                        				(dijkstras gsize g src nadj) 
                        				{p=nvLTE {gsize=gsize} (getVal v)}) 
                        		(length psv) = True
 

dijkstras_correctness (S len) g src v psv spsv nadj {weight} {ops}
  = (l5_sp {cl=runDijkstras col} 
           (correctness col nadj lemma2_ih base_stm)) 
           v 
           (allExp g src nodes dist) psv spsv
   ...
\end{lstlisting}


The type of \inl{dijkstras_correctness} states that, given a input graph \inl{g} and the source node \inl{src}, for any node \inl{v} in the graph and a shortest path named \inl{psv} from \inl{src} to \inl{v} in the graph \inl{g}(specified by the input \inl{spsv}), the distance value from \inl{src} to \inl{v} calculated by running \inl{dijkstras} function (from our implementation in Section \inl{implementation}) is equal to the length of the shortest path \inl{psv}, i.e., is the minimum distance from \inl{src} to \inl{v}. The function \inl{indexN} uses the value carried by the node \inl{v} to index its distance value from a given \inl{Vect}, which is the \inl{Vect} of distance values returned by \inl{`dijkstras gsize g src nadj'} in the type of \inl{dijkstras_correctness} provided above. 
\\

To implement the proof \inl{dijkstras_correctness}, we first construct the first \inl{Column} \inl{cl} in the matrix representation of Dijkstra's, in the same way the \inl{dijkstras} function (Section \ref{first_layer}). Then given two proofs \inl{lemma2_ih} and \inl{base_stms}, which state that the properties specified by \inl{neDInfPath} and \inl{l5_stms} hold for \inl{cl}, we apply \inl{correctness} on \inl{cl, lemma2_ih}, and \inl{base_stms} to obtain the proof \inl{`l5_stms (runDijkstras cl)'}, which states that the properties specified by \inl{l5_stms} hold for the \inl{Column} calculated by \inl{`runDijkstras cl'}. Hence the fourth statement in \inl{l5_stms}, which is \inl{expDistIsDelta}(Section \ref{lemma5V}), also holds for \inl{`runDijkstras cl'}. By applying \inl{`expDistIsDelta (runDijkstras cl)'} on the input node \inl{v} and its shortest path \inl{psv}, we prove that for any node \inl{v} in the input graph \inl{g}, the distance value of \inl{v} stored in the \inl{Vect} calculated by \inl{`dijkstras gsize g src nadj'} is indeed the minimum distance value from \inl{src} to \inl{v}, which verifies the correctness of Dijkstra's algorithm. 
\\

Unfortunately, the implementation of \inl{lemma2_ih} and \inl{base_stms} is still incomplete, since the current definitions of the \inl{mkdists} and \inl{mknodes} functions are hard to work with in implementing the proofs. However the incompleteness does not largely affect the validity of our verification program, as this issue can be resolved by providing a better, easy-to-approach definitions for \inl{mkdists} and \inl{mknodes}, and we are confident to provide the full implementation if granted more time. 
\\





