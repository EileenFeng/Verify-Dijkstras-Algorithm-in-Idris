
\section{Related Work}
The increasing importance of Dijkstra's algorithm in many real-world applications has raised an interest on verifying it's implementation. Robin Mange and Jonathan Kuhn provide a project that verifies a Java implementation of Dijkstra's algorithm with the Jahob verification system in their report on efficient proving of Java programs[3]. Although we failed to obtain the concrete implementation of this work, the report demonstrates the verification process. Function behaviors are specified with preconditions, frame conditions, and postconditions, and Jahob allows programmers to provide these specifications in high-level logic(HOL), which reduces the problem of program verification to the validity of HOL formulas. 
% Existing work on verifying Dijkstra's algorithm is relatively limited, and few resources are found for the verification of Bellman-Ford algorithm. Robin Mange and Jonathan Kuhn demonstrates an implementation that verifies Dijkstra's algorithm with the Jahob verification system in their report on efficient proving of Java implementations[3]. Although few resource has been found on the concrete implementation of this work, the report illustrates that as Jahob allows programmers to provide specification of their function's behaviors in high-level logic(HOL), program verification can be reduced to the problem of the validity of HOL formulas. 
\\

Klasen et. al. from the University of Koblenz and Landau verifies Dijkstra's algorithm with the KeY system[1], an interactive theorem prover for Java. Concrete implementations of Dijstra's algorithm with different variants are provided, and all of them are written in Java. Simiarly to the work by Mange and Kuhn, the verification process in the work by Klasen involves describing the behavior of each function with pre- and postconditions and modifies clause. Loop invariants are specified to support the verification. A function is then examine as correct by the KeY systemm, with respect to its behavior specifications, if the postconditions specified hold after execution. A similar implementation is provided by Jean-Christophe Filliâtre, a senior researcher from the National Center for Scientific Research(CNRS), which verifies Dijkstra's implementation with Why3, a deductive program verification platform that relies on external theorem provers[4][5]. All works presented above are largely dependent on theorem proving systems, however our work relies on a significantly smaller trusted code base. Most proofs in our work will be implemented from scratches, and considerable amount of details on verification will be presented explicitly.
\\

In spite of the popularity of Bellman-Ford algorithm in network applications, no resources are found on verifying implementations of Bellman-Ford algorithm.

% Jean-Christophe Filliâtre, a senior researcher from the National Center for Scientific Research(CNRS), offers an implementation of Dijkstra's algorithm along with its verification in Why3, a deductive program verification platform that relies on external theorem provers[4][5]. Both verifications above are largely dependent on theorem proving systems. Unlike Filliâtre and Klasen et. al., our work relies on a significantly smaller trusted code base, indicating that considerable amount of proofs will be presented explicitly in our implementation rather than provided by external theorem provers.
