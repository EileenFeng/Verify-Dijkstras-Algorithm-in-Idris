\documentclass[12pt, oneside]{article}
\usepackage[utf8]{inputenc} % allow utf-8 input
%\usepackage[T1][fontize=12]{fontenc}    % use 8-bit T1 fonts
\usepackage{hyperref}       % hyperlinks
\usepackage{url}            % simple URL typesetting
\usepackage{booktabs}       % professional-quality tables
\usepackage{amsfonts}       % blackboard math symbols
\usepackage{nicefrac}       % compact symbols for 1/2, etc.
\usepackage{microtype}      % microtypography
\usepackage{lipsum}
\title{\Large{\textbf{Shortest Path Algorithms Verification with Idris}} \\ \textbf{\large{Thesis Outline}}}
\author{Yazhe Feng}
%\date{}							% Activate to display a given date or no date

\begin{document}
\maketitle
\section{Introduction}
\begin{itemize}
	\item State the problem: Dijkstra's and Bellman-Ford algorithms are widely used in real-life applications, however existing work for verifying shortest path algorithms are limited
	\item Contributions
	\item Structure of thesis
\end{itemize}

\section{Motivation}
\begin{itemize}
	\item Show the importance of verification in assisting people to ensure the behavior of programs
	\item Given the significance of Dijkstra's and Bellman-Ford in real-life applications, it is important to verify their correctness
	\item Present algorithm verification as a programming issue: with programming languages such as Idris, we can implement programs that verify other programs
\end{itemize}

\section{Background}
\begin{itemize}
	\item Introduce Idris programming languages and its features (dependent types etc.)
	\item Brief introduction of Dijkstra's and Bellman-Ford algorithms
\end{itemize}

\section{High-level Contribution}
\begin{itemize}
	\item Pseudocode for Dijkstra's and Bellman-Ford, followed up by overview of the implementation of both algorithms in Idris 
	\item Theoretical proofs of Dijktra's and Bellman-Ford based on the Idris implementation 
\end{itemize}

\section{Low-level Contribution}
\begin{itemize}
	\item Details on the implementation of Dijkstra's and Bellman-Ford in Idris: data structures, functions, program control flow, and assumptions
	\item Assumptions of the proof (expected qualities of input data etc.)
	\item Overall structure of our verification, and details on the proof: provide type signatures and interpretation of types for key lemmas
\end{itemize}

\section{Discussion}
\begin{itemize}
	\item Progress on verification versus initial expectations
	\item Things that have been tried but failed to achieve
	\item Reflection on the process of program verification: useful tips and advices etc. 
\end{itemize}

\section{Related Work}
\begin{itemize}
	\item Present existing work of Dijkstra's verification
		\begin{enumerate}
			\item Verifying Dijkstra's Algorithm with KeY: \url{https://kola.opus.hbz-nrw.de/opus45-kola/frontdoor/deliver/index/docId/420/file/DA_KLASEN.pdf}
			\item Verifying Dijkstra’s algorithm in Jahob: \url{http://lara.epfl.ch/w/_media/dijkstra.pdf}
		\end{enumerate}
	\item Comparison with existing works
		\begin{enumerate}
			\item Verifying in Idris versus other languages or proof assistance
			\item Different approaches taken during verification
		\end{enumerate}
\end{itemize}

\section{Conclusion}
\begin{itemize}
	\item Briefly conclude motivations, goals, and contributions
\end{itemize}



\end{document}


