\documentclass[11pt, oneside]{article}   	% use "amsart" instead of "article" for AMSLaTeX format
\usepackage{geometry}
\geometry{
  top = 23mm,
  left = 18mm, 
  right = 18mm,
  bottom = 25mm,
}
\renewcommand{\baselinestretch}{1}
\newcommand\tab[1][1cm]{\hspace*{#1}}
\newcommand\ftab[1][5cm]{\hspace*{#1}}
\newcommand\tsp[1][0.2cm]{\hspace*{#1}}
\usepackage[]{algorithm2e}
\geometry{letterpaper}
\usepackage{graphicx}
\usepackage{amssymb}
\usepackage{amsthm}
\usepackage{titlesec}
\titlespacing*{\section}
{0pt}{1.5\baselineskip}{1\baselineskip}
\theoremstyle{definition}
\newtheorem{definition}{Definition}[section]

%\usepackage{parskip}

\graphicspath{ {images/} }

\newtheorem*{theorem}{Theorem}
\newtheorem*{lemma}{Lemma}
%SetFonts

\title{Dijkstra's Algorithm Verification}
\author{Yazhe Feng}
%\date{}							% Activate to display a given date or no date

\begin{document}
\maketitle

\section{Dijkstra's Algorithm}

\subsection{Pseudocode}
Given input graph $g$ and source node $s$ with types:
\\\\
  \tab g : Graph gsize weight\\
  \tab s : Node gsize
\\\\
We denote $(u, v)$ as an edge from node $u$ to $v$, $weight(u, v)$ as the weight of edge $(u, v)$. We define $unexplored$ as the list of unexplored nodes, and $dist$ as the list storing distance from $s$ to each node $n \in g$
\\\\
\tab (initially $unexplored$ contains all nodes in graph $g$)\\
\tab $unexplored : List (Node \tsp gsize)$\\
\tab $unexplored = \{v : v \in g\}$
\\\\
\tab (node value is used to index $dist$, initially distance of all nodes are infinity except 
\\ \tab the source node)\\
\tab $dist : List \tsp weight$ \\
\tab $dist[s] = 0, dist[a] = infinity, \forall a \in g, a \neq s$
\\\\
The Dijkstra's Algorithm runs as follows: 
\\\\
\texttt{
  \tab while (unexplored is not Nil) 
  \tab$\{$ \\
  \tab\tab (At the $k^{th}$ iteration of the while loop)                                          \\
  \tab\tab choose $u \in unexplored$ s.t.$\forall u' \in unexplored, dist[u] \leq dist[u']$     \\
  \tab\tab let $unexplored'$ be the list after removing $u$ from $unexplored$                    \\
  \tab\tab for($\forall v \in g$ s.t.$(u, v) \in g$) $\{$                                 \\  
  \tab\tab\tab (At the $p^{th}$ iteration of this for loop)                                \\
  \tab\tab\tab  if($dist[u] + weight(u, v) < dist[v]$) $\{$                              \\
  \tab\tab\tab\tab  let $dist' = dist$ with $dist'[v] = dist[u] + weight(u, v)$          \\
  \tab\tab\tab $\}$ \\ 
  \tab\tab\tab input the new $dist'$ to the $(p+1)^{th}$ iteration of the for loop \\
  \tab\tab $\}$ \\
  \tab\tab input the new $unexplored'$ and $dist'$ to the $(k+1)^{th}$ iteration of the while loop \\
  \tab $\}$
}

\subsection{Assumptions}
\begin{enumerate}
  \item Weight of edges are non-negative
  \item All nodes $n$ and edge $e$ are valid: $n, e \in g$
\end{enumerate}



\section{Definition}
\theoremstyle{definition}
\begin{definition}\textbf{Path}\\
\textit{(We adopt the definition of $path$ presented in the \texttt{Discrete Mathematics with Applications} book by \texttt{SUSANNA S. EPP}.)}
\\\\
A path from node $v$ to $w$ is a finite alternating sequence of adjacent vertices and edges of G, which does not contain any repeated edge or vertex. A path from $v$ to $w$ has the form: 
\begin{center}
 $ve_0v_0e_1v_2....v_{n-1}e_nw$ 
\end{center}
where $e_i$ is an edge in $g$ with endpoints $v_{i-1}, v_i$. We denote the set of paths from $v$ to $w$ as $path(v, w)$.
\end{definition}
\tab
\begin{definition}\textbf{Length of Path} \\
The length of a path $p = ve_0v_0e_1v_2....v_{n-1}e_nw$ is the sum of the weights of all edges in $p$. We write: 
\begin{center}
  $length(p) = \sum weight(e_i), \forall e_i \in p$. 
\end{center} 
\end{definition}
\tab
\begin{definition}\textbf{Shortest Path}\\
Denote $\Delta(s, v)$ as the shortest path from $s$ to $v$. $\Delta(s, v)$ must fulfills: 
\begin{center}
$\Delta(s, v) \in path(s, v)$ \tsp and \tsp $\forall p' \in path(s, v)$, $length(\Delta(s, v)) \leq length(p')$
\end{center}
\end{definition}

\section{Proof of Correctness}
\subsection{Proof of Termination}
The inner for loop is guaranteed to terminate as the algorithm goes through each adjacent node exactly once. As the size of list \texttt{unexplored} decreases by one during each iteration of the while loop, the algorithm is guaranteed to terminate. 


\subsection{Proof of Correctness}
Given graph $g$ and source node $s$, $dist$ stores the distance value from $s$ to all nodes in $g$ calculated by the Dijkstra's algorithm, $dist[v]$ gives the corresponding distance value of $v$ from $s$. Denote $explored$ as the list of nodes in $g$ but not in $unexplored$, i.e., $explored$ stored all nodes whose neighbors have been updated by the algorithm. We also denote $\delta(v)$ as the distance value of the shortest path from $s$ to node $v \in g$. 
\newpage
\begin{lemma}
[1] During each iteration of the algorithm, forall node $v \in explored$, we have:
\begin{enumerate}
  \item $\delta(v) < \delta(v')$, $\forall v' \in unexplored$.
  \item $dist[v] = \delta(v)$
\end{enumerate}
\end{lemma}
\begin{proof}
We will prove this by inducting on the number of iterations. 
\\\\
Let P(n) be: During the $n^{th}$ iteration of the algorithm, $n > 1$, for all nodes $v \in explored$ of length, (1) $\delta(v) < \delta(v')$, $\forall v' \in unexplored, v \neq v'$, and (2) $\delta(v) = dist[v]$. 
\\\\
\textbf{Base Case}: 
We shall show P(1) holds. 
\\
Based on the algorithm, during the first iteration, the node with minimum distance value is the source node $s$ with $dist[s] = 0$. Hence during the first iteration, only $s$ is removed from $unexplored$ and added to $explored$. Since all edge weights are non-negative, then $dist[s] = \delta(s)$ and $\delta(s) < \delta(v')$ $\forall v' \neq s$. Hence P(1) holds. 
\\\\
\textbf{Inductive Hypothesis}: Suppose P(k) holds forall $k > 1$. That is, during the $k^{th}$ iteration of the algorithm for all $k > 1$, forall $v \in explored$, (1) $\delta(v) < \delta(v')$, $\forall v' \in unexplored, v \neq v'$, and (2) $\delta(v) = dist[v]$. 
\\\\
\textbf{Inductive Step}: We shall show P(k+1) holds. 
\\ 
Suppose $v$ is the node added into $explored$ during the $(k+1)^{th}$ iteration. We need to show (1) $\delta(v) < \delta(v')$, $\forall v' \in unexplored, v' \neq v$, and (2) $dist[v] = \delta(v)$. 
\begin{enumerate}
\item $\delta(v) \leq \delta(v')$, $\forall v' \in unexplored, v' \neq v$
\\\\
We will prove (1) by contradiction. Suppose there exists $w \in unexplored$, such that $\delta(w) < \delta(v)$. Since $w \in unexplored$ and $v \in explored$, then based on the algorithm, $dist[v] < dist[w]$ holds for the $(k+1)^{th}$ iteration. Since $\delta(v) \leq dist[v]$ and $dist[v] < dist[w]$, then $\delta(v) < dist[w]$ holds for the $(k+1)^{th}$ iteration ([a]). 
\\
Assume $w'$ is the node just before $w$ in $\Delta(s,w)$(Definition 2.3). Then we have:
\\\\
\ftab $\delta(w) = dist[w'] + weight(w', w)$ 
\\\\
Since $\delta(w) < \delta(v)$, then: 
\\\\
\ftab $\delta(w) < \delta(v)$ \\
\ftab $dist[w'] + weight(w', w) < \delta(v)$ \\
\ftab $dist[w'] < \delta(v)$ \\
\\\\
Since $dist[w'] < \delta(v)$ and $\delta(v) \leq dist[v]$, then $dist[w'] < dist[v]$. Thus based on the algorithm, the node $w'$ must have been explored before $v$, i.e. $w' \in explored$. Since $w'$ has an edge $(w', w)$ to $w$, then the algorithm must have compared $(dist[w'] + weight(w', w))$ with the current $dist[w]$ before the $(k+1)^{th}$ iteration and chose $dist[w]$. Thus it must be $(dist[w'] + weight(w', w)) \geq dist[w]$, i.e. $\delta(w) \geq dist[w]$. Since $\delta(v) > \delta(w)$ and $\delta(w) \geq dist[w]$, then $\delta(v) > dist[w]$, which contradicts with [a]. Hence by the principle of prove by contradiction, (1) holds for the $(k+1)^{th}$ iteration. 
\\
\item $dist[v] = \delta(v)$
\\\\
Suppose $dist[v]$ is associates with path $p \in path(s, v)$ during the $(k+1)^{th}$ iteration, and assume the shortest path from $s$ to $v$ is some path $p' \in path(s, v)$ different than $p$, $length(p') = \delta(v) < dist[v]$([b]). Suppose $v'$ is the node just before $v$ in $p'$. 
\\\\
\ftab $\delta(v) = dist[v'] + weight(v', v)$ \\
\\\\
Since all edge weights are non-negative, then $dist[v'] < \delta(v)$. Based on (1), since $\delta(v) < \delta(w) \forall w \in unexplored$, then $v'$ must be in $explored$. Since $v'$ is in $explored$ and has an edge to $v$, then the algorithm must have compared $dist[v'] + weight(v', v)$ to the current $dist[v]$ and chose $dist[v]$. Hence it must be $dist[v'] + weight(v', v) \geq dist[v]$, i.e. $\delta(v) \geq dist[v]$, which contradicts with [b]. Hence by the principle of prove by contradiction, $p$ is the shortest path from $s$ to $v$, and that $dist[v] = \delta(v)$. 
\end{enumerate}
\end{proof}
\tab\\
\begin{proof}\textbf{Prove of Correctness}
\\\\
By applying Lemma (1) to each iteration of the algorithm, we obtained that for all nodes $n$ in the explored list, $dist[n]$ is indeed the shortest path distance value from source $s$ to $n$, hence Dijkstra's algorithm indeed calculates the shortest path distance value from the source $s$ to each node $n \in g$. 
\end{proof}
\end{document}


